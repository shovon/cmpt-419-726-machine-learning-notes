\documentclass{tufte-handout}

\title{Machine Learning}

\author[Salehen Shovon Rahman]{Salehen Shovon Rahman}

%\date{28 March 2010} % without \date command, current date is supplied

%\geometry{showframe} % display margins for debugging page layout

\usepackage{graphicx} % allow embedded images
  \setkeys{Gin}{width=\linewidth,totalheight=\textheight,keepaspectratio}
  \graphicspath{{graphics/}} % set of paths to search for images
\usepackage{amsmath}  % extended mathematics
\usepackage{booktabs} % book-quality tables
\usepackage{units}    % non-stacked fractions and better unit spacing
\usepackage{multicol} % multiple column layout facilities
\usepackage{lipsum}   % filler text
\usepackage{fancyvrb} % extended verbatim environments
  \fvset{fontsize=\normalsize}% default font size for fancy-verbatim environments

% Standardize command font styles and environments
\newcommand{\doccmd}[1]{\texttt{\textbackslash#1}}% command name -- adds backslash automatically
\newcommand{\docopt}[1]{\ensuremath{\langle}\textrm{\textit{#1}}\ensuremath{\rangle}}% optional command argument
\newcommand{\docarg}[1]{\textrm{\textit{#1}}}% (required) command argument
\newcommand{\docenv}[1]{\textsf{#1}}% environment name
\newcommand{\docpkg}[1]{\texttt{#1}}% package name
\newcommand{\doccls}[1]{\texttt{#1}}% document class name
\newcommand{\docclsopt}[1]{\texttt{#1}}% document class option name
\newenvironment{docspec}{\begin{quote}\noindent}{\end{quote}}% command specification environment

\begin{document}

\maketitle% this prints the handout title, author, and date

\begin{abstract}
\noindent
Personal notes regarding machine learning.
\end{abstract}

%\printclassoptions

\section{Polynomial Curve Fitting}\label{sec:polynomial-curve-fitting}

\newthought{Here's an example machine learning problem}: try to find the best
polynomial that can potentially fit a set of data points, and have it be fit as
best as possible. This is known as the polynomial curve fitting problem, and
it's a supervised regression learning problem.

\begin{marginfigure}
  \includegraphics[width=\linewidth]{curvefitting.png}
  \caption{An example data set where we want to fit a polynomial curve into}
\end{marginfigure}

\subsection{The Problem}\label{sec:headings}

\newthought{Suppose} we are given a training set of $N$ observations, $(x_{1},
x_{2}, \ldots, x_{N})$ and $(t_{1}, t_{2}, \ldots, x_{N})$, $x_{i}, t_{i} \in
\mathbb{R}$. We want to find a polynomial $y(x)$ that fits these data the best.

Let's start out by defining a $y(x, \mathbf{w})$.

\begin{equation}
  y(x, \mathbf{w}) = w_0 + w_1x + w_2x^2 + \ldots + w_Mx^M =
  \sum\limits_{i = 1}^M(w_ix^i)
\end{equation}

How do we measure success? Or, a better question, for what values of the
coefficients $\mathbf{w}$ will yield the best results? To answer that, we define
an error function $E$.

\begin{equation}
  E(\mathbf{w}) = \frac{1}{2}\sum\limits_{i = 1}^N(y(x_i, \mathbf{w}) - t_i)^2
\end{equation}

We then use the $\arg\min\limits_{x}f(x)$ function to find the value for the
parameter that yields the lowest value in a given function.

\begin{equation}
  \mathbf{w}^* = \arg \min\limits_{\mathbf{w}}E(\mathbf{w})
\end{equation}

$\min\limits_{x} f(x)$ finds the lowest possible value for the
expression $f(x)$, while $\arg\min\limits_{x} f(x)$ finds the value for $x$
where $f(x)$ would be the lowest.

So, in other words, we want to find a $\mathbf{w}$ such that $E(\mathbf{w})$ is
the lowest among the set of all possible values of $\mathbf{w}$.

\newthought{Except}, the attempt at finding a value $\mathbf{w}^*$ such that
$E(\mathbf{w}^*) = 0$ can become problematic.

Earlier, we mentioned that we had an initial set of training data. However, for
most cases, when trying to fit the polynomial such that $E(\mathbf{w}^*) = 0$,
for the training set, we risk having it so that when a test data set is
introduced, the error function yields a high value. This is known as
\textit{overfitting}.

\begin{marginfigure}
  \includegraphics[width=\linewidth]{overfitting.png}
  \caption{As we can see, the first few polynomials of degree $N < 9$ fit the
    data fine, even when test data is introduced to the training set, but misses
    the mark entirely when $N = 9$. This is the result of overfitting.}
\end{marginfigure}

In the end of the day, although we want the curve to fit the data as best as
possible, we also want a \textit{genralization} derived from the given training.

\newthought{But first}, before we go ahead with finding a good gneralization,
for convenience, instead of just using the error function $E$, we use the
root-mean-square (RMS) error function, defined by

\begin{equation}
  E_{\text{RMS}} = \sqrt{2E(\mathbf{w}^*)/N}
\end{equation}

According to the text book, \textit{Pattern Recognition and Machine Learning},
the reason why we are using RMS, is the following:

\begin{quote}
  [The] division by $N$ allows us to compare different sizes of data sets
  on an equal footing, and the square root ensures that $E_{\text{RMS}}$ is
  measured on the same scale (and in the same units) as the target variable $t$.
  (p. 7)
\end{quote}

Now, to actually tune our $\mathbf{w}$ for better generalization, we can split
our training data into two sets: training set and validation set. In the case of
finding the polynomial, the training set can be used to find each $w_i \in
\mathbf{w}$, and the validation set is used to optimize the complexity, which
can be represented by $M$ (the size of $\mathbf{w}$), or a $\lambda$, which
will be discussed later.

There are several techniques to used to control overfitting.

\newthought{The first technique} to avoid overfitting is being cross-validation.

Here, we group the data into separate sets. We first ``train'' our parameters to
a union of all the separated set, while leaving one out. Then we optimize by
including the one we initially excluded. Afterwards, we ``train'' again with a
new union of our sets, while leaving yet another one out, but including the one
that we initially left out, all the way until no sets are left to leave out.

\newthought{And then}, there's regularization for controlling over-fitting.

Notice how the oscillation increases as $M$ increases? This is because the
magnitudes of the coefficients in $\mathbf{w}$ as $M$ increases.

\begin{marginfigure}
  \includegraphics[width=\linewidth]{oscillations.png}
  \caption{The oscillation is far wider for $M = 9$ than $M = 0$}
\end{marginfigure}

Hello, Universe

\section{Bernoulli trials}

While both the \docopt{number} and \docopt{offset} arguments are optional, they
must be provided in order.  To adjust the vertical position of the sidenote
while leaving the sidenote number alone, use the following syntax:
\begin{docspec}
  \doccmd{sidenote[][\docopt{offset}]\{\docarg{Sidenote text.}\}}
\end{docspec}
The empty brackets tell the \Verb|\sidenote| command to use the default
sidenote number.

If you \emph{only} want to change the sidenote number, however, you may
completely omit the \docopt{offset} argument:
\begin{docspec}
  \doccmd{sidenote[\docopt{number}]\{\docarg{Sidenote text.}\}}
\end{docspec}

The \Verb|\marginnote| command has a similar \docarg{offset} argument:
\begin{docspec}
  \doccmd{marginnote[\docopt{offset}]\{\docarg{Margin note text.}\}}
\end{docspec}

\subsection{References}
References are placed alongside their citations as sidenotes,
as well.  This can be accomplished using the normal \Verb|\cite|
command.\sidenote{The first paragraph of this document includes a citation.}

The complete list of references may also be printed automatically by using
the \Verb|\bibliography| command.  (See the end of this document for an
example.)  If you do not want to print a bibliography at the end of your
document, use the \Verb|\nobibliography| command in its place.

To enter multiple citations at one location,\cite{Tufte2006,Tufte1990} you can
provide a list of keys separated by commas and the same optional vertical
offset argument: \Verb|\cite{Tufte2006,Tufte1990}|.
\begin{docspec}
  \doccmd{cite[\docopt{offset}]\{\docarg{bibkey1,bibkey2,\ldots}\}}
\end{docspec}

\section{Figures and Tables}\label{sec:figures-and-tables}
Images and graphics play an integral role in Tufte's work.
In addition to the standard \docenv{figure} and \docenv{tabular} environments,
this style provides special figure and table environments for full-width
floats.

Full page--width figures and tables may be placed in \docenv{figure*} or
\docenv{table*} environments.  To place figures or tables in the margin,
use the \docenv{marginfigure} or \docenv{margintable} environments as follows
(see figure~\ref{fig:marginfig}):

\begin{marginfigure}%
  \includegraphics[width=\linewidth]{helix}
  \caption{This is a margin figure.  The helix is defined by
    $x = \cos(2\pi z)$, $y = \sin(2\pi z)$, and $z = [0, 2.7]$.  The figure was
    drawn using Asymptote (\url{http://asymptote.sf.net/}).}
  \label{fig:marginfig}
\end{marginfigure}
\begin{Verbatim}
\begin{marginfigure}
  \includegraphics{helix}
  \caption{This is a margin figure.}
\end{marginfigure}
\end{Verbatim}

The \docenv{marginfigure} and \docenv{margintable} environments accept an optional parameter \docopt{offset} that adjusts the vertical position of the figure or table.  See the ``\nameref{sec:sidenotes}'' section above for examples.  The specifications are:
\begin{docspec}
  \doccmd{begin\{marginfigure\}[\docopt{offset}]}\\
  \qquad\ldots\\
  \doccmd{end\{marginfigure\}}\\
  \mbox{}\\
  \doccmd{begin\{margintable\}[\docopt{offset}]}\\
  \qquad\ldots\\
  \doccmd{end\{margintable\}}\\
\end{docspec}

Figure~\ref{fig:fullfig} is an example of the \Verb|figure*|
environment and figure~\ref{fig:textfig} is an example of the normal
\Verb|figure| environment.

\begin{figure*}[h]
  \includegraphics[width=\linewidth]{sine.pdf}%
  \caption{This graph shows $y = \sin x$ from about $x = [-10, 10]$.
  \emph{Notice that this figure takes up the full page width.}}%
  \label{fig:fullfig}%
\end{figure*}

\begin{figure}
  \includegraphics{hilbertcurves.pdf}
%  \checkparity This is an \pageparity\ page.%
  \caption{Hilbert curves of various degrees $n$.
  \emph{Notice that this figure only takes up the main textblock width.}}
  \label{fig:textfig}
  %\zsavepos{pos:textfig}
  \setfloatalignment{b}
\end{figure}

Table~\ref{tab:normaltab} shows table created with the \docpkg{booktabs}
package.  Notice the lack of vertical rules---they serve only to clutter
the table's data.

\begin{table}[ht]
  \centering
  \fontfamily{ppl}\selectfont
  \begin{tabular}{ll}
    \toprule
    Margin & Length \\
    \midrule
    Paper width & \unit[8\nicefrac{1}{2}]{inches} \\
    Paper height & \unit[11]{inches} \\
    Textblock width & \unit[6\nicefrac{1}{2}]{inches} \\
    Textblock/sidenote gutter & \unit[\nicefrac{3}{8}]{inches} \\
    Sidenote width & \unit[2]{inches} \\
    \bottomrule
  \end{tabular}
  \caption{Here are the dimensions of the various margins used in the Tufte-handout class.}
  \label{tab:normaltab}
  %\zsavepos{pos:normaltab}
\end{table}

\section{Full-width text blocks}

In addition to the new float types, there is a \docenv{fullwidth}
environment that stretches across the main text block and the sidenotes
area.

\begin{Verbatim}
\begin{fullwidth}
Lorem ipsum dolor sit amet...
\end{fullwidth}
\end{Verbatim}

\begin{fullwidth}
\small\itshape\lipsum[1]
\end{fullwidth}

\section{Typography}\label{sec:typography}

\subsection{Typefaces}\label{sec:typefaces}
If the Palatino, \textsf{Helvetica}, and \texttt{Bera Mono} typefaces are installed, this style
will use them automatically.  Otherwise, we'll fall back on the Computer Modern
typefaces.

\subsection{Letterspacing}\label{sec:letterspacing}
This document class includes two new commands and some improvements on
existing commands for letterspacing.

When setting strings of \allcaps{ALL CAPS} or \smallcaps{small caps}, the
letter\-spacing---that is, the spacing between the letters---should be
increased slightly.\cite{Bringhurst2005}  The \Verb|\allcaps| command has proper letterspacing for
strings of \allcaps{FULL CAPITAL LETTERS}, and the \Verb|\smallcaps| command
has letterspacing for \smallcaps{small capital letters}.  These commands
will also automatically convert the case of the text to upper- or
lowercase, respectively.

The \Verb|\textsc| command has also been redefined to include
letterspacing.  The case of the \Verb|\textsc| argument is left as is,
however.  This allows one to use both uppercase and lowercase letters:
\textsc{The Initial Letters Of The Words In This Sentence Are Capitalized.}



\section{Installation}\label{sec:installation}
To install the Tufte-\LaTeX\ classes, simply drop the
following files into the same directory as your \texttt{.tex}
file:
\begin{quote}
  \ttfamily
  tufte-book.cls\\
  tufte-common.def\\
  tufte-handout.cls\\
  tufte.bst
\end{quote}

% TODO add instructions for installing it globally



\section{More Documentation}\label{sec:more-doc}
For more documentation on the Tufte-\LaTeX{} document classes (including commands not
mentioned in this handout), please see the sample book.

\section{Support}\label{sec:support}

The website for the Tufte-\LaTeX\ packages is located at
\url{https://github.com/Tufte-LaTeX/tufte-latex}.  On our website, you'll find
links to our \smallcaps{svn} repository, mailing lists, bug tracker, and documentation.

\bibliography{sample-handout}
\bibliographystyle{plainnat}



\end{document}
